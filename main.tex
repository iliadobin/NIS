\documentclass{letask}
\usepackage{textcomp}	%типографские знаки
\usepackage{tempora} %Times New Roman alike
\usepackage{setspace}
% \usepackage[14pt]{extsizes}
\usepackage{anyfontsize}
%\полуторный интервал
\onehalfspacing
\linespread{1.5}


\begin{document}
\begin{titlepage}
\center % Center everything on the page
 
%----------------------------------------------------------------------------------------
%	HEADING SECTIONS
%----------------------------------------------------------------------------------------

\textsc{\LARGE Высшая\\[-0.2cm]Школа Экономики\\[0.1cm]\large Национальный Исследовательский Университет}\\[1.5cm] % Name of your university/college
\textsc{\Large Научный Исследовательский семинар}\\[0.1cm] % Major heading such as course name
\textsc{\large Обзор литературы}\\[0.5cm] % Minor heading such as course title

%----------------------------------------------------------------------------------------
%	TITLE SECTION
%----------------------------------------------------------------------------------------

\HRule
\\[0.4cm]
{ \huge \bfseries Высокочастотная торговля\\[0.2cm]
и ее влияние на эффективность и баланс сил на биржах}
\\[0.6cm] % Title of your document
\HRule
\\[1.5cm]


 
%----------------------------------------------------------------------------------------
%	AUTHOR SECTION
%----------------------------------------------------------------------------------------

\begin{minipage}{0.4\textwidth}
	\begin{flushleft} \large
		\textsf{Студент}
		
		Добин \\[-0.15cm] Илья  \textsc{} \\[-0.15cm]
		БЭАД223
	\end{flushleft}
\end{minipage}
~
\begin{minipage}{0.4\textwidth}
	\begin{flushright} \large
		\textsf{Преподаватель}
		
	Степанченко\\[-0.15cm] Дмитрий\\[-0.15cm]
		\textsc{} % Supervisor's Name
	\end{flushright}
\end{minipage}

\begin{bottompar}
	
	{\large 10 Декабря 2023}

\end{bottompar}
\vfill % Fill the rest of the page with whitespace

\end{titlepage}	
 
\begin{flushright}
	\textit{}\\
	\textit{}\\
\end{flushright}

\fontsize{14}{14}\selectfont 

Высокочастотная торговля (High-Frequency Trading, HFT) представляет собой феномен в мире финансов, который существенно изменил традиционные методы проведения биржевых операций. Эта форма торговли характеризуется высокой частотой совершения сделок, измеряемой в миллисекундах, и использованием мощных компьютерных алгоритмов для принятия решений. Основной целью высокочастотной торговли является получение прибыли от краткосрочных изменений в ценах финансовых инструментов.

Исходя из своей природы, высокочастотная торговля стремится к автоматизации процессов торговли и максимизации скорости реакции на рыночные события. Это достигается благодаря использованию передовых технологий, включая быстрые интернет-соединения, специализированные аппаратные ускорители и сложные математические модели.

Однако, несмотря на свою технологическую совершенность, высокочастотная торговля вызывает широкий спектр вопросов и обсуждений, особенно в контексте ее воздействия на эффективность и баланс сил на финансовых рынках. Этот обзор направлен на рассмотрение ключевых аспектов высокочастотной торговли и ее роли в современной биржевой деятельности.

Начнем с обсуждения влияния ВЧТ на рыночную ликвидность. Рыночная ликвидность отражает способность актива быстро и с минимальными потерями превратиться в наличные средства на открытом рынке. Это свойство важно для инвесторов и трейдеров, поскольку высокая рыночная ликвидность обычно означает, что актив легко купить или продать без существенного влияния на его цену. Рынки с хорошей ликвидностью часто более устойчивы к внезапным колебаниям и потрясениям. Увеличение объема сообщений, направляемых на биржу алгоритмическими трейдерами, приводит к интенсивным дискуссиям среди исследователей и практиков. Рената Карковска в своей статье "Does high-frequency trading actually improve market liquidity?" в ходе сравнения регрессионных моделей, связывающих различные показатели рыночной ликвидности с высокочастотной активностью на одном и том же наборе данных, обнаруживает, что для некоторых моделей увеличение высокочастотной активности способствует улучшению рыночной ликвидности, тогда как для других наблюдается противоположный эффект. Показано, что увеличение числа высокочастотных сделок положительно влияет на ликвидность, в то время как увеличение числа высокочастотных заказов может привести к ее снижению. Это подчеркивает неоднозначность влияния ВЧТ на рынок и связанную с этим сложность анализа.

Юрай Грушка в статье "The Dynamics of Liquidity on the German Stock Market Under the Influence of HFT" изучает взаимосвязь рыночной ликвидности акций, торгуемых на Франкфуртской фондовой бирже, и активностью высокочастотной торговли. Для определения этих соотношений используются эконометрические методы панельной регрессии. Результаты данной работы указывают, что алгоритмическая торговля оказывает большое положительное влияние на рынок, например рост ликвидности за счет снижения спредов. Автор приходит к схожим выводам в другой своей статье, написанной в соавторстве с Дагмар Линнертовой "Liquidity of the European Stock Markets Under the Influence of HFT". В ней они исследуют взаимосвязь между рыночной ликвидностью фьючерсов, торгуемых на бирже EUREX, и деятельностью HFT на европейских рынках производных инструментов. Для выявления данных зависимостей используются эконометрические методы анализа временных рядов. Результаты модели оказались смешанными, и их влияние сильно зависит от метода измерения волатильности. Волатильность представляет собой меру изменчивости цен. Она отражает степень колебаний или разброса ценовых изменений для данного актива или рынка в определенный период времени. Однако, несмотря на смешанные результаты, воздействие высокочастотной торговли на ликвидность рынка оценивается как положительное. Неоднозначность результатов, возможно, объясняется использованием приближенных значений для измерения ликвидности из-за ограниченной доступности публичной информации о рассматриваемом рынке.

Статья Имена Бен Аммара "High-frequency trading and stock liquidity: An intraday analysis" утверждает, что высокочастотная торговля (HFT) оказывает существенное влияние на интрадневную ликвидность акций CAC40 на бирже Euronext.То есть он изучает как изменяется ликвидность в течение одного торгового дня на бирже. Наблюдается формирование характерных интрадневных паттернов в спредах и глубине рынка. Спред - это разница между ценами покупки (BID) и продажи (ASK) одного и того же финансового инструмента. Сужение спреда обычно свидетельствует о более высокой ликвидности, тогда как расширение может быть признаком низкой ликвидности или волатильности на рынке. Автор утверждает, что высокий уровень отмены ордеров в периоды повышенного спроса на ликвидность связан с высокочастотными трейдерами, это влияет на структуру спредов и может свидетельствовать о стремлении адаптировать свои позиции к изменяющимся условиям рынка. Анализ, проведенный с использованием метода обобщенных моментов, подтверждает, что увеличение активности HFT связано с уменьшением спредов и увеличением глубины рынка. Положительный эффект HFT на ликвидность обусловлен, в основном, снижением затрат на неблагоприятный отбор в условиях асимметричной информации на рынке.

Продолжая тему ликвидности, рассмотрим статью Панагиотиса Анагностидиса "Liquidity commonality and high frequency trading: Evidence from the French stock market", в которой он, используя набор данных о заказах и сделках на французском фондовом рынке, демонстрирует, что ликвидность, предоставляемая высокочастотными трейдерами, менее разнообразна по сравнению с традиционными трейдерами. Это может свидетельствовать о том, что алгоритмы ВЧТ являются источником систематических ликвидационных потрясений, то есть значительных изменений в уровне ликвидности на рынке, которые выражаются в резком уменьшении доступности кредитов, скачках волатильности и так далее. В статье "Profitability and liquidity provision of HFTs during large price shocks: Does relative tick size matter?" Ямада Масахиро исследует прибыль и предоставление ликвидности высокочастотными трейдерами во время крупных ценовых скачков на финансовых рынках. Ранее проведенные исследования предоставляли противоречивые доказательства в отношении того, предоставляют ли HFT ликвидность или забирают ее в таких событиях. Эмпирические результаты данной статьи указывают на то, что относительный размер тика имеет значение: HFT предоставляют ликвидность, когда скачок является идиосинкратическим и относительный размер тика велик, но в этом случае они не извлекают прибыль от торговли. В среднем HFT могут извлекать прибыль, торгуя агрессивно против не-HFT и в направлении скачков для акций с небольшим относительным размером тика. (Тик - это минимальное изменение цены, которое может произойти на рынке)

Используя данные Chi-X, Лаура Мальцениесе в статье "High frequency trading and comovement in financial markets" обнаруживает, что HFT приводит к значительному увеличению сопряженности (comovement) в доходах и ликвидности. Примерно треть увеличения сопряженности доходов объясняется более быстрым распространением рыночной информации. Оставшуюся две трети автор связывает с коррелированными торговыми стратегиями HFT. Увеличение сопряженности ликвидности согласуется с тем, что поставщики ликвидности HFT лучше могут мониторить другие акции и соответственно корректировать свое предложение ликвидности. Полученные результаты предполагают механизм, через который HFT влияет на стоимость капитала.

До этого в статьях в основном рассматривались рынки, на которых высокочастотные трейдеры были крупными игроками. Джумхур Экинджи в своей статье "High-frequency trading and market quality: The case of a “slightly exposed” market" решил рассмотреть случай, когда ВЧТ не является заметной фигурой на рынке. Автор рассматривает 30 акций синих фишек на развивающемся рынке Borsa Istanbul. Несмотря на низкую долю в общем объеме деятельности, ВЧТ имеет заметные последствия. Например, предоставление ликвидности со стороны не-HFT трейдеров значительно уменьшается в присутствии HFT. При этом активность HFT не оказывает влияния на уровень волатильности рынка. Это может поддерживать утверждение о том, что, несмотря на свои заметные эффекты на ликвидность, HFT не сильно влияет на степень колебаний цен. Также HFT демонстрирует способность генерировать прибыль как в дни с положительным, так и с отрицательным изменением цен. Это указывает на то, что HFT может успешно функционировать в различных рыночных условиях. Эти выводы вызывают обеспокоенность в отношении ВЧТ и свидетельствуют о том, что потенциальные внешние факторы характерны не только для рынков с преобладанием ВЧТ.

В статье "Algorithmic trading and market quality: International evidence of the impact of errors in colocation dates" Майкла Эйткена рассматривается влияние ошибок в датах размещения оборудования  на эффективность рынка. 
Даты размещения оборудования это моменты времени, когда участники рынка размещают свои торговые серверы и другое оборудование физически близко к инфраструктуре биржи. Это сделано с целью минимизации задержек в передаче данных и обеспечения максимальной быстродействия при исполнении торговых операций. Это становится особенно важным для высокочастотных трейдеров, для которых даже малейшая задержка в получении информации и исполнении сделок может иметь существенное значение. Такие участники рынка стремятся минимизировать время передачи данных, чтобы иметь преимущество в совершении операций. Статья обсуждает различные аспекты связанные с датами размещения оборудования, такие как их влияние на ликвидность рынка, возможность реакции на появление высокочастотных трейдеров. Ошибки в colocation dates могут оказывать существенное влияние на результаты исследований. Это подчеркивает важность точности данных при изучении воздействия алгоритмической торговли на рыночное качество.

Теперь перейдем к статье Маттео Аквилина "Sharks in the dark: Quantifying HFT dark pool latency arbitrage
", рассматривающей ситуацию на темных пулах. "Темные пулы" (dark pools) представляют собой альтернативные торговые площадки, где совершаются торговые операции, но без публичной видимости стандартных биржевых торгов. Они получили название "темные", потому что торги на этих площадках не отражаются в стандартных биржевых стаканах (торговых книгах) и не доступны общественности до их исполнения. Основная идея темных пулов заключается в том, чтобы предоставить трейдерам возможность совершать крупные сделки без воздействия на цену актива на открытом рынке. Такие сделки могут быть скрыты от общественности, что особенно важно при работе с большими объемами акций. В статье исследуется проблема "stale reference pricing" и предоставление ликвидности в темных пулах с использованием внутренних данных по участникам, предоставленных регуляторами. Авторы демонстрируют, что существенная часть торговли в темных пулах является устаревшей, что накладывает значительные издержки на пассивных участников темных пулов. Согласно этим издержкам, высокочастотные трейдеры почти никогда не предоставляют ликвидность в темных пулах, а, наоборот, часто потребляют ликвидность, особенно чтобы воспользоваться устаревшими референсными ценами. В заключение, авторы показывают, что вмешательства в дизайн рынка, включающие случайное выбор времени исполнения в темных пулах, успешно противостоят арбитражу задержки, защищая пассивных предоставителей ликвидности.

Статья Витора Леоне "High frequency trading, price discovery and market efficiency in the FTSE100" сфокусирована на FTSE100 и рассматривает роль ВЧТ в выявлении цен и эффективности рынка. (Financial Times Stock Exchange 100 Index) — это индекс фондового рынка Великобритании, который отражает стоимость акций 100 крупнейших компаний, котирующихся на Лондонской фондовой бирже (LSE). Выводы указывают на то, что ВЧТ способствует обнаружению цен и эффективности рынка, особенно в краткосрочной временной перспективе. Тем не менее, существует предел времени, после которого рыночная случайность начинает преобладать, что имеет важные последствия для арбитражных стратегий.

В статье Флавио Баццана "How does HFT activity impact market volatility and the bid-ask spread after an exogenous shock? An empirical analysis on S\&P 500 ETF" проводится эмпирический анализ инфра-секундных данных по SPDR S\&P 500 ETF с целью объяснения того, как деятельность высокочастотной торговли – как агрессивная, так и пассивная – влияет на волатильность рынка и разрыв спроса и предложения до и после внешнего воздействия (в данном случае, выборов президента США в 2016 году). Используя данные SPDR S\&P 500 ETF в качестве приближенной модели для рынка в обычные дни торговли и в дни с высоким объемом торговли, авторы показывают, что HFT в среднем оказывает беспокоящее воздействие в основном в обычные дни торговли, в то время как в дни с высоким объемом торговли оно, кажется, оказывает стабилизирующее воздействие, балансируя как волатильность, так и разрыв спроса и предложения. То есть HFT в целом оказывает более нейтральное воздействие на волатильность рынка и разрыв спроса и предложения, чем его отдельные агрессивные и пассивные компоненты. На самом деле, агрессивное HFT оказывает последовательное негативное воздействие, которое, в среднем, увеличивает как волатильность, так и разрыв спроса и предложения, тогда как пассивное HFT проявляет положительное воздействие, которое, в среднем, уменьшает как волатильность, так и разрыв спроса и предложения.

Анализ ряда статей посвященных высокочастотной торговле выявил многообразие воздействия этой торговой стратегии на финансовые рынки. Высокочастотная торговля стала неотъемлемой частью современных рынков, однако ее влияние на эффективность и баланс сил остается предметом активных дискуссий и исследований.

В контексте ликвидности, некоторые исследования указывают на то, что HFT может способствовать улучшению рыночной ликвидности, снижая спреды и обеспечивая более эффективное обнаружение цен. Однако, другие работы подчеркивают двусмысленность влияния HFT на ликвидность, особенно в темных пулах, где неконтролируемая торговля может привести к потреблению ликвидности, а не ее предоставлению.

Важным аспектом является также воздействие HFT на волатильность рынка и бид-аск спред. В некоторых случаях HFT может усиливать волатильность, особенно в периоды обычной торговли объемом. Тем не менее, обнаружено, что этот вид торговли может также стабилизировать рынок в условиях больших объемов торгов, демонстрируя комплексное воздействие на рыночную динамику.

Статьи также подчеркивают важность регулирования и надзора в контексте HFT, особенно в темных пулах, где несвоевременная торговля и устаревшие базовые цены могут создавать неравные условия для участников рынка. Исследования в этой области продолжают эволюционировать, и для понимания полной картины воздействия HFT на биржи требуется дальнейшая работа.

\section{Список литературы}
Renata Karkowska, Andrzej Palczewski (2023). Does high-frequency trading actually improve market liquidity?

Cumhur Ekinci, Oğuz Ersan (2022). High-frequency trading and market quality: The case of a “slightly exposed” market.

Panagiotis Anagnostidis, Patrice Fontaine (2020). Liquidity commonality and high frequency trading: Evidence from the French stock market.

Vitor Leone, Frank Kwabi (2019). High frequency trading, price discovery and market efficiency in the FTSE100.

Imen Ben Ammar, Slaheddine Hellara, Imen Ghadhab (2020). High-frequency trading and stock liquidity: An intraday analysis.

Juraj Hruska, Dagmar Linnertova (2015). Liquidity of the European Stock Markets Under the Influence of HFT.

Michael Aitken, Douglas Cumming, Feng Zhan (2023). Algorithmic trading and market quality: International evidence of the impact of errors in colocation dates.

Masahiro Yamada (2022). Profitability and liquidity provision of HFTs during large price shocks: Does relative tick size matter?

Matteo Aquilina, Sean Foley, Peter O'Neill, Thomas Ruf (2024). Sharks in the dark: Quantifying HFT dark pool latency arbitrage.

Juraj Hruška (2016). Dynamics of Liquidity on German Stock Market Under the Influence of HFT.

Flavio Bazzana, Andrea Collini (2020). How does HFT activity impact market volatility and the bid-ask spread after an exogenous shock? An empirical analysis on S&P 500 ETF.

Laura Malceniece, Kārlis Malcenieks, Tālis J. Putniņš (2019). High frequency trading and comovement in financial markets.
\end{document}
